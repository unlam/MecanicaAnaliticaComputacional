\documentclass[11pt, spanish, a4paper, twoside]{article}
% LTeX: language=es-AR

\input{../figurasLaTeX/introUNLaM}


\begin{document}

\begin{center}
  % \textsc{\large Mecánica general}\\
  \textsc{\large Coordenadas generalizadas | Ligaduras | Energías cinética y potencial}
\end{center}
\noindent
Los problemas marcados con (*) tienen alguna dificultad adicional, no dude en consultar.

\begin{enumerate}

		%\item \begin{minipage}[t][7.1cm]{0.75\textwidth}
%\textbf{Péndulo simple} [Marion (e) ex. 7.2] \\
%Obtenga la ecuación diferencial que describe la dinámica de una pesa que ``pendulea'' en el extremo de una cuerda.
%\begin{enumerate}
%	\item Si el péndulo oscila dentro del plano \(\hat{x}, \hat{y}\).
%		¿En que sistema de coordenadas resolverá el problema? 
%		¿Cuál coordenada es relevante para describir la dinámica? 
%	\item Enumere las aproximaciones del modelo de péndulo que resolverá que lo diferencian de uno que puede armar en el laboratorio.
%	\item Calcule la energía potencial de la pesa en el campo gravitatorio.
%	¿Para qué sirve eso?
%	Las fuerzas que surgen de un campo son fácilmente calculables usando que \(\vec{F} = - \vec{\nabla} V\), es decir, \emph{la fuerza es igual al negativo del gradiente del potencial}.
%	\item Escriba la 2.a ley de Newton para la coordenada relevante.
%	\item Resuelva la ecuación de la dinámica y obtenga la frecuencia de oscilación.
%\end{enumerate}
%\end{minipage}
%\begin{minipage}[c][0cm][t]{0.2\textwidth}
%%	\includegraphics[width=\textwidth]{marion_fig7_1}
%%	\includegraphics[width=\textwidth]{\detokenize{pénduloHorizontal}}
%	\begin{tikzpicture}[scale= 1.0]
%	\draw [arrows=-latex] (-1,2) -> (-1,1) node [above=15, right=2] {\(\vec{g}\)}; % g vertical
%		\draw [ultra thick] (-1.5,3) -- (2,3);
%		\fill [pattern = north east lines] (-1.5,3) rectangle (2,3.2); % techo
%		\draw [dashed] (0,3) -- (0,-.25);	% vertical
%		\draw [thick] (0,3) -- +(-60:3) node[midway,above,right=2] {\(\ell\)};	% inclinada +:relativa, -60 grados, longitud 3
%		\shade [ball color=black!80] ($(0,3)+(-60:3)$) circle(0.25) node [] {\color{white} $m$};
%    \draw [arrows=-latex] (0,.4) -> (1.25,.4) node [midway, above] {\( \psi \)}; % desplazamiento horizontal
%		\draw [arrows=-latex] (0,0) arc [start angle=-90, end angle=-65, radius=3] node [below=12, left=8] {\( \varphi \)};
%	\end{tikzpicture}
%\end{minipage}
%%\vspace{-2.5cm}
%

%\subsection*{Fuerza como gradiente de un potencial}
%% Vizcaino carpeta
%\item Una partícula esta sometida a una fuerza \(\vec{F}(x)= \left( -k x + \frac{a}{x^3}\right) \hat{x}\)
%\begin{enumerate}
%    \item Hallar el potencial \(U(x)\) y graficarlo.
%    \item Discutir los tipos de movimientos posibles.
%    \item Determinar las posiciones de equilibrio estable y encuentre la solución general de la dinámica de la partícula, \(\vec{r}(t)\).
%    \item Interpretar el movimiento en el límite \(E^2 \gg k a\). ¿Cuanto vale el periodo de las oscilaciones?
%    \item Ídem. anterior en el límite \(E^2 \rightarrow k a\) con \(E^2 > k a\).
%\end{enumerate}



%\item \begin{minipage}[t][3cm]{0.7\textwidth}
%\textbf{Péndulo ideal rígido} [Marion ex. 7.2]\\
%Escriba y resuelva la ecuación que describe la dinámica de un péndulo de longitud $\ell$ en presencia de un campo gravitatorio de constante $g$. Discuta todas las aproximaciones que realiza.
%\end{minipage}
%\begin{minipage}[c][1cm][t]{0.25\textwidth}
%%	\includegraphics[width=0.75\textwidth]{marion_fig7_1}
%%	\includegraphics[width=\textwidth]{\detokenize{pénduloHorizontal}}
%	\begin{tikzpicture}
%	%\begin{tikzpicture}[scale= 1.2]
%  	\draw [arrows=- triangle 45] (-1,2) -> (-1,1) node [above=15, right=2] {\(\vec{g}\)}; % g vertical
%		\draw [ultra thick] (-1.5,3) -- (2.5,3);
%		\fill [pattern = north east lines] (-1.5,3) rectangle (2.5,3.2); % techo
%		\draw [dashed] (0,3) -- (0,0);	% vertical
%		\draw [thick] (0,3) -- +(-60:3) node[midway,above,right=2] {\(\ell\)};	% inclinada +:relativa, -60 grados, longitud 3
%		% \draw [thick] (0,3) -- +(-60:3) node[midway,above,sloped] {\(l\)};	% inclinada +:relativa, -60 grados, longitud 3
%    \fill (0,3)+(-60:3) circle [radius=0.25] node [text=white] { \( \mathrm{m} \) }; % masa izq
%    \draw [thin, arrows=- triangle 45] (0,.4) -> (1.25,.4) node [midway, above] {\( \psi \)}; % desplazamiento horizontal
%    % \draw [thin, arrows=- triangle 45] (0,.4) -> (1.25,.4) node [above=8, left=8] {\( \psi \)}; % desplazamiento horizontal
%		\draw [thin, arrows=- triangle 45] (0,0) arc [start angle=-90, end angle=-65, radius=3] node [below=12, left=8] {\( \varphi \)};
%	\end{tikzpicture}
%\end{minipage}




\item
	\begin{minipage}[t][2.5cm]{0.7\textwidth}
		\textbf{Péndulo con punto de suspensión libre} [Landau \S5 ej. 2]\\
		La partícula de masa \(m_2\) pende de una barra rígida de longitud \(\ell\) de masa despreciable.
		En su otro extremo hay un dispositivo de masa \(m_1\) enhebrado en una barra rígida horizontal y que se mueve libremente a lo largo de su eje \(\hat{x}\).
		El dispositivo permite que la barra que pende de él forme con la vertical cualquier ángulo \(\varphi\). 
	\end{minipage}
	\begin{minipage}[c][2cm][t]{0.3\textwidth}
		\includegraphics[width=0.75\textwidth]{landauS52_fig2.png}
	\end{minipage}
	\begin{enumerate}
		\item Tras determinar las coordenadas generalizadas, escriba la posición de las partículas en función de ellas.
		\item Calcule las velocidades de las partículas.
		\item Con estas velocidades, calcule la energía cinética, \(T\), y potencial gravitatoria, \(V\), de cada partícula.  
		\item Calcule ahora \(T\) y \(V\) usando las funciones que toman por parámetros las masas y posiciones de las partículas. Verifique que obtiene el mismo resultado en menos pasos.
		\item Realice substituciones en las expresiones de \(T\) y \(V\) del punto anterior para inmovilizar la partícula de masa \(m_1\). Verifique que recupera las expresiones que corresponden a las de un péndulo rígido ideal.
	\end{enumerate}



\item
	\begin{minipage}[t][3.7cm]{0.7\textwidth}
		\textbf{Péndulo doble} [Landau \S5 ej. 1]\\
		Una barra rígida de longitud \(\ell_1\) tiene una masa despreciable respecto a la de la partícula de masa \(m_1\) fija a su extremo.
		A su vez de esta última pende otra barra rígida, de longitud \(\ell_2\) que en su extremo tiene otra partícula de masa \(m_2\), también mucho mayor que aquella de la barra.
	
		Para cada uno de los siguientes puntos escriba en un cuaderno Jupyter titulado con su apellido una o varias celdas de código separadas entre sí por otras conteniendo un texto indicando de qué punto se trata.
	\end{minipage}
	\begin{minipage}[c][2cm][t]{0.3\textwidth}
		\includegraphics[width=0.75\textwidth]{landauS52_fig1.png}
	\end{minipage}

	\begin{enumerate}
		\item Obtenga la energía cinética, \(T\), y potencial, \(V\), en función de las coordenadas de la figura.
		
		Resultado:\\
		\(
			T_{traslaci\acute{o}n} = \dfrac{\ell_{1}^{2} m_{1} \dot{\varphi}_{1}^{2}}{2} + \dfrac{m_{2} \left(\ell_{1}^{2} \dot{\varphi}_{1}^{2} + 2 \ell_{1} \ell_{2} \cos{\left(\varphi_{1} - \varphi_{2} \right)} \dot{\varphi}_{1} \dot{\varphi}_{2} + \ell_{2}^{2} \dot{\varphi}_{2}^{2}\right)}{2}\\
			V_{gravitatoria} = - g \left(\ell_{1} m_{1} \cos{\left(\varphi_{1} \right)} + \ell_{1} m_{2} \cos{\left(\varphi_{1} \right)} + \ell_{2} m_{2} \cos{\left(\varphi_{2} \right)}\right)
		\)
		\item Establezca \(m_1 = 0\), \(\varphi_1 = \varphi_2 = \varphi\) y \(\ell_1 = \ell_2 = \frac{\ell}{2}\) a través de la función de substitución de SymPy. Verifique que se obtiene el \(T\) y \(V\) de un único péndulo rígido ideal.
	\end{enumerate}
	% Ayuda: \( \cos(\alpha \pm \beta) = \cos{ \alpha} \cos{ \beta \mp \sin \alpha} \sin{ \beta } \)



\item
	\begin{minipage}[t][7.1cm]{0.55\textwidth}
		(*) \textbf{Péndulo con punto de suspensión en rotación} [Marion (e) ex. 7.5] [Landau \S5 ej. 3]\\

		Una partícula de masa \(m\) pende de una barra rígida de longitud \(b\).
		El punto de suspensión engarzado en un aro de radio \(a\) dispuesto verticalmente rota respecta a su centro con una frecuencia \(\omega\) constante.
		Se asume que todas las posiciones se encuentran en un único plano bidimensional y que la masa de la barra rígida tiene masa despreciable frente a \(m\).

		Calcule la energía cinética, \(T\) y potencial, \(V\) de la partícula con masa \(m\).

		Resultado:\\
		\(
			T_{traslaci\acute{o}n} = \dfrac{m \left(a^{2} \omega^{2} - 2 a b \omega \sin{\left(\omega t - \theta \right)} \dot{\theta} + b^{2} \dot{\theta}^{2}\right)}{2}\\
			V_{gravitatoria} = g m \left(a \sin{\left(\omega t \right)} - b \cos{\left(\theta \right)}\right)
		\)
	\end{minipage}
	\begin{minipage}[c][1.5cm][t]{0.5\textwidth}
		\includegraphics[width=0.75\textwidth]{marion_fig7_3.png}
		% \includegraphics[width=0.75\textwidth]{landauS52_fig3.png}
	\end{minipage}



\item
	\begin{minipage}[t][7cm]{0.65\textwidth}
		(*) \textbf{Pesas acopladas rotando en torno a eje} [Landau \S5 ej. 4]\\

		La pieza con \(m_2\) se desplaza sobre un eje vertical, y todo el sistema gira con una velocidad angular constante \(\Omega\) en torno a ese eje.
		Está unida por barras de longitud \(a\) y masa despreciable a otras pesas de masa \(m_1\).
		A su vez estas penden de sendas barras idénticas del punto fijo, \(A\), que describen un ángulo variable de apertura respecto al eje \(\theta\).

		Calcule la energía cinética para cada una de las tres masas y exprese en la forma más compacta posible la del sistema en su conjunto.
		Haga lo propio con la energía potencial.

		Resultado:\\
		\(
			T_{traslaci\acute{o}n} = a^{2} \left( m_{1} \left( \Omega^{2} \sin^{2}{\left(\theta \right)} + \dot{\theta}^{2}\right) + 2 m_{2} \sin^{2}{\left(\theta \right)} \dot{\theta}^{2} \right) \\
			V_{gravitatoria} = - 2 a g \left(m_{1} + m_{2}\right) \cos{\left(\theta \right)}
		\)
	\end{minipage}
	\begin{minipage}[c][1cm][t]{0.35\textwidth}
		\includegraphics[width=0.75\textwidth]{landauS52_fig4.png}
	\end{minipage}

	\begin{center}
		\includegraphics[width=0.35\textwidth]{rotantesAcopladas1.png}
		\includegraphics[width=0.35\textwidth]{rotantesAcopladas2.png}
	\end{center}

	En estas ilustraciones indicamos el punto \(A\), de donde tenemos \emph{agarrado} al sistema, con la \emph{pelota de arriba}.
	Todo gira en torno al eje rosa con velocidad angular CONSTANTE $\Omega$.
	Por lo tanto, las dos partículas de los laterales, las de masa $m_1$, rotan entrando y saliendo del plano de la pantalla (imagen de abajo).
	Esto es equivalente a pensar que el plano celeste de la imagen rota completo sobre el eje rosa.
	
	La pieza de masa $m_2$ es un dispositivo pasante (un buje) que puede ir para arriba y abajo sin rozamiento sobre el eje vertical (rosa).
	Si la pieza de abajo sube, todos los ángulos cambian lo mismo, porque las longitudes de las barras naranjas son todas iguales.
	
	\begin{center}
		\includegraphics[width=0.35\textwidth]{rotantesAcopladas3.png}
		\includegraphics[width=0.35\textwidth]{rotantesAcopladas4.png}
	\end{center}
	

\end{enumerate}
\end{document}
