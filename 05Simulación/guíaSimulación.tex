\documentclass[11pt, spanish, a4paper, twoside]{article}
% LTeX: language=es-AR

\input{../figurasLaTeX/introUNLaM}


\begin{document}
\begin{center}
  % \textsc{\large Mecánica general}\\
  \textsc{\large Simulación de la dinámica | Resolución numérica de la ecuación de Euler-Lagrange}
\end{center}

\noindent
%De poder resolver estos problemas en forma autónoma puede asumir que adquirió los conocimientos mínimos sobre los temas abordados en la semana.
%No dude en consultar a docentes y compañeros si no puede terminarlos.
En los siguientes problemas resolverá numericamente cada ecuación de Euler-Lagrange que corresponda a cada coordenada generalizada.
Graficando tales soluciones, en el rango de tiempos y con las condiciones iniciales indicadas, estará simulando la dinámica de tales sistemas.\\
La aceleracion gravitatoria tiene por magnitud \(|\vec{g}| = \SI{9.81}{\metre\per\second\squared}\).\\
Los problemas marcados con (*) tienen alguna dificultad adicional, no dude en consultar.


\begin{enumerate}


\item 
\begin{minipage}[t][2.5cm]{0.7\textwidth}
\textbf{Máquina de Atwood simple}\\
Rango de tiempo \(t = \SIrange{0}{10}{\second}\).
Parámetros físicos y condiciones iniciales:\\
\(\ell_\mathrm{cuerda} > \SI{150}{\metre}\), 
\(R_{\mathrm{polea}\,1} = \SI{0.5}{\metre}\), \\ 
\(m_1 = \SI{8}{\kilo\gram}\), 
\(m_2 = \SI{1}{\kilo\gram}\), 
\(m_p = \SI{4}{\kilo\gram}\), \\
\(x(t=0) = \SI{25}{\metre}\), 
\(\dot{x}(t=0) = -\SI{10}{\metre\per\second}\).
\end{minipage}
\begin{minipage}[c][2cm][t]{0.3\textwidth}
	%\hspace{0.5cm}
	%\includegraphics[width=0.6\textwidth]{marion_fig2_1a}
	\begin{tikzpicture}
	% define constants for tikz
	\def \radius {1.0};
	\def \deltax {1.0};
	\def \boxwidth {\radius/ 2.5};
	\def \boxAheight {-1.5};
	\def \boxBheight {-2.5};
	\def \pende {\radius + \boxwidth + 0.2};
	\def \pendeLeft {-\radius - \boxwidth - 0.2};

	% circle at 0,0
	\draw [thick] (0,0) circle (\radius);
	\filldraw (0,0) circle (1pt);

	% dashed lines from 0.1 at each side of the circle
	\draw [dashed] (0.2,0) -- ({\radius + \deltax},0);
	\draw [dashed] (-0.2,0) -- ({-\radius - \deltax},0);

	% box centred at (-\radius, \boxAheight)
	\draw [thick] (-\radius -\boxwidth, \boxAheight -\boxwidth) rectangle ({-\radius + \boxwidth},\boxAheight + \boxwidth) node [midway] {\(m_1\)}; 
	\draw [thick] ( \radius -\boxwidth, \boxBheight -\boxwidth) rectangle ({ \radius + \boxwidth}, \boxBheight + \boxwidth) node [midway] {\(m_2\)};

	% draw the line connecting the two boxes to the circle
	\draw [very thick] (-\radius, \boxAheight + \boxwidth) -- (-\radius,0);
	\draw [very thick] ( \radius, \boxBheight + \boxwidth) -- (\radius,0); 

	% draw dashed lines for y coordinates from horizontal lines to the height of middle of the boxes
	\draw [-Latex] (\pendeLeft, 0) -- (\pendeLeft, \boxAheight) node [midway, left] {\(y_1\)};
	\draw [-Latex] ( \pende, 0) -- ( \pende, \boxBheight) node [midway, right] {\(y_2\)};

\end{tikzpicture}
\end{minipage}



\item
	a) \textbf{Péndulo rígido ideal} [Marion (english) ex. 7.2] \\
	b) \textbf{Péndulo con punto de suspensión libre} [Landau \S5 ej. 2]\\ 
	c) \textbf{Péndulo doble} [Landau \S5 ej. 1] 
\begin{tasks}(3)
	\task \input{figs/pendulum.tikz}
	\task \includegraphics[height=0.24\textwidth]{landauS52_fig2}
	\task \includegraphics[height=0.24\textwidth]{landauS52_fig1}
\end{tasks}
Rango de tiempo \(t = \SIrange{0}{10}{\second}\).
Parámetros físicos y condiciones iniciales: 
\begin{enumerate}
	\item \(m = \SI{3}{\kilo\gram}\), 
				\(\ell = \SI{2}{\metre}\), 
				\(\varphi (t=0) = \frac{\pi}{4}\), \(\dot{\varphi} (t=0) = 0\).
	\item \(m_1 = \SI{3}{\kilo\gram}\), \(m_2 = \SI{1}{\kilo\gram}\),   
				\(\ell = \SI{2}{\metre}\), 
				\(x(t=0) = \SI{1}{\metre}\), \(\dot{x} (t=0) = \SI{0.5}{\metre\per\second} \),
				\(\phi (t=0) = \frac{\pi}{8}\), \(\dot{\phi} (t=0) = 0\).
	\item \(m_1 = \SI{3}{\kilo\gram}\), \(m_2 = \SI{1}{\kilo\gram}\),
				\(\ell_1 = \SI{1}{\metre}\), \(\ell_2 = \SI{1}{\metre}\),\\ 
				\(\phi_1 (t=0) = \frac{\pi}{8}\), \(\dot{\phi}_1 (t=0) = 0\), 
				\(\phi_2 (t=0) = \frac{\pi}{4}\), \(\dot{\phi}_2 (t=0) = -\frac{\pi}{16} \si{\per\second}\).
\end{enumerate}


\item 
	\begin{minipage}[t][3cm]{0.7\textwidth}
	\textbf{Péndulo de pesas deslizantes y acopladas}\\ 
	Rango de tiempo \(t = \SIrange{0}{10}{\second}\).
	Parámetros físicos y condiciones iniciales:\\
	\(m_1 = m_2 = m = \SI{2}{\kilo\gram}\), \(l = \SI{2}{\metre}\), \(\theta(t=0) = \frac{\pi}{4}\), \(\dot{\theta}(t=0) = 0\).
	\end{minipage}
	\begin{minipage}[c][2cm][t]{0.3\textwidth}
		\input{figs/slidingWeights.tikz}
		%\hspace{0.5cm}
  	%\includegraphics[width=0.6\textwidth]{fcen1-004}
	\end{minipage}



%\item \begin{minipage}[t][3.5cm]{0.6\textwidth}
%\textbf{Taylor ej. 7.2}
%Encuentre las ecuaciones de Euler-Lagrange para una partícula moviendose en dos dimensiones usando coordenadas polares.
%Asuma la presencia de una energía potencial \(V(r,\phi)\).
%\end{minipage}
%\begin{minipage}[c][1em][t]{0.35\textwidth}
%	\hspace{0.5cm}
%   \includegraphics[width=0.75\textwidth]{taylorFig7-1}
%\end{minipage}
%




\newpage
\item
\begin{minipage}[t][2cm]{0.65\textwidth}
(*) \textbf{Maquina de Atwood compuesta} [Marion (english) ex. 7.8]\\ 
Rango de tiempo \(t = \SIrange{0}{5}{\second}\).
Parámetros físicos y condiciones iniciales:\\
\(\ell_\text{superior} = \SI{15}{\metre}\), 
\(R_{\text{polea sup}} = \SI{0.5}{\metre}\), 
\(\ell_\text{inferior} = \SI{15}{\metre}\), 
\(R_{\text{polea inf}} = \SI{0.5}{\metre}\),\\ 
\(m_1 = \SI{1}{\kilo\gram}\),
\(m_2 = \SI{2}{\kilo\gram}\),
\(m_3 = \SI{3}{\kilo\gram}\),
\(M_{\text{polea sup}} = \SI{4}{\kilo\gram}\),
\(M_{\text{polea inf}} = \SI{4}{\kilo\gram}\),\\
\(y(t=0) = \SI{1}{\metre}\), \(\dot{y}_1(t=0) = 0\),
\(y_2(t=0) = \SI{2}{\metre}\), \(\dot{y}_2(t=0) = 0\)
\end{minipage}
\begin{minipage}[c][3cm][t]{0.3\textwidth}
	\input{figs/compoundAtwood.tikz}
	% \includegraphics[width=\textwidth]{marion_fig7_6}
\end{minipage}




\end{enumerate}
\end{document}
