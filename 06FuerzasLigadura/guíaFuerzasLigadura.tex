\documentclass[11pt, spanish, a4paper, twoside]{article}
% LTeX: language=es-AR

\input{../figurasLaTeX/introUNLaM}

\begin{document}
\begin{center}
  % \textsc{\large Mecánica general}\\
  \textsc{\large Fuerzas de ligadura | Multiplicadores de Lagrange} 
\end{center}

\begin{enumerate}

	\item
	% \begin{minipage}[t][3.5cm]{0.7\textwidth}
	\begin{minipage}[t][0cm]{0.7\textwidth}
		\textbf{Péndulo rígido ideal}\\
		Calcule la tensión de la cuerda con el método de multiplicadores de Lagrange.
		La restricción es que la pesa se mantiene siempre en \(\vec{r} = \ell \hat{\rho}\), ergo la función que expresa esto es \(f(\rho) = \rho - \ell = 0\).
	\end{minipage}
	\begin{minipage}[c][0cm][t]{0.25\textwidth}
		\input{figs/pendulum.tikz}
	\end{minipage}



	\item 
	\begin{minipage}[t][3cm]{0.47\textwidth}
	\textbf{Cilindro que rueda por un plano inclinado} [Marion (e) ex. 7.5]\\
	% English \textbf{Marion ejemplo 6.5 y ejemplo 7.9} Disco rodando en un plano inclinado.
		\begin{enumerate}
			\item Encuentre las ecuaciones de movimiento, 
			\item la aceleración angular,
			\item y la fuerzas de ligadura. 
		\end{enumerate}
	\end{minipage}
	\begin{minipage}[c][2cm][t]{0.3\textwidth}
		\includegraphics[width=\textwidth]{marion_fig6_7_dva}
		% \includegraphics[width=\textwidth]{marion_fig6_7}
	\end{minipage}

	
	\item
	\begin{minipage}[t][7.5cm]{0.67\textwidth}
	\textbf{Doble máquina de Atwood} [Marion (e) ej. 7.8 y 7-37]\\
	Utilice el método de multiplicadores de Lagrange para encontrar las ecuaciones de movimiento y las tensiones de las cuerdas.
	\begin{enumerate}
		\item Verifique que obtiene las mismas aceleraciones generalizadas que se habían obtenido sin usar multiplicadores de Lagrange.
		Resultado:\\[5 pt]
		\(
			\ddot{y}_{1} = 
			\dfrac{2 g \left(2 m_{1} m_{2} + 2 m_{1} m_{3} + m_{1} m_{p} - 8 m_{2} m_{3} - 3 m_{2} m_{p} - 3 m_{3} m_{p} - m_{p}^{2}\right)}{4 m_{1} m_{2} + 4 m_{1} m_{3} + 2 m_{1} m_{p} + 16 m_{2} m_{3} + 8 m_{2} m_{p} + 8 m_{3} m_{p} + 3 m_{p}^{2}}\\[5 pt]
			\ddot{y}_{2} = 
			\dfrac{2 g \left(4 m_{1} + m_{p}\right) \left(m_{2} - m_{3}\right)}{4 m_{1} m_{2} + 4 m_{1} m_{3} + 2 m_{1} m_{p} + 16 m_{2} m_{3} + 8 m_{2} m_{p} + 8 m_{3} m_{p} + 3 m_{p}^{2}}
		\)
		\item Obtenga las tensiones de ambas cuerdas.
			Resultado:\\[5 pt]
			\(
				Q_{1} = \dfrac{g \left(- 32 m_{1} m_{2} m_{3} - 12 m_{1} m_{2} m_{p} - 12 m_{1} m_{3} m_{p} - 4 m_{1} m_{p}^{2} - 8 m_{2} m_{3} m_{p} - 3 m_{2} m_{p}^{2} - 3 m_{3} m_{p}^{2} - m_{p}^{3}\right)}{4 m_{1} m_{2} + 4 m_{1} m_{3} + 2 m_{1} m_{p} + 16 m_{2} m_{3} + 8 m_{2} m_{p} + 8 m_{3} m_{p} + 3 m_{p}^{2}}
				% Q_{1} = 
				% \dfrac{g \left(32 m_{1} m_{2} m_{3} + 12 m_{1} m_{2} m_{p} + 12 m_{1} m_{3} m_{p} + 4 m_{1} m_{p}^{2} + 8 m_{2} m_{3} m_{p} + 3 m_{2} m_{p}^{2} + 3 m_{3} m_{p}^{2} + m_{p}^{3}\right)}{4 m_{1} m_{2} + 4 m_{1} m_{3} + 2 m_{1} m_{p} + 16 m_{2} m_{3} + 8 m_{2} m_{p} + 8 m_{3} m_{p} + 3 m_{p}^{2}}
				\\[5 pt]
				Q_{2} = \dfrac{g m_{3} \left(- 16 m_{1} m_{2} - 4 m_{1} m_{p} - 4 m_{2} m_{p} - m_{p}^{2}\right)}{4 m_{1} m_{2} + 4 m_{1} m_{3} + 2 m_{1} m_{p} + 16 m_{2} m_{3} + 8 m_{2} m_{p} + 8 m_{3} m_{p} + 3 m_{p}^{2}}
				% Q_{2} = 
				% \dfrac{g m_{3} \left(16 m_{1} m_{2} + 4 m_{1} m_{p} + 4 m_{2} m_{p} + m_{p}^{2}\right)}{4 m_{1} m_{2} + 4 m_{1} m_{3} + 2 m_{1} m_{p} + 16 m_{2} m_{3} + 8 m_{2} m_{p} + 8 m_{3} m_{p} + 3 m_{p}^{2}}
			\)
		\end{enumerate}
	\end{minipage}
	\begin{minipage}[c][2cm][t]{0.3\textwidth}
		\input{figs/compoundAtwood.tikz}
		% \includegraphics[width=\textwidth]{marion_fig7_6}
	\end{minipage}

	% \newpage

	\item
	\begin{minipage}[t][7.5cm]{0.65\textwidth}
		\textbf{Pesos enlazados por una cuerda} [Taylor 7.50]\\
		Una partícula de masa \(m\), situada sobre una mesa horizontal sin fricción, está unida mediante una cuerda ideal de longitud \(\ell\) a otra partícula de masa \(M\).
		La cuerda pasa por un orificio practicado en la mesa, el cual no presenta rozamiento.
		La segunda pesa pende vertical con una distancia a la mesa \(y = \ell - \rho\), función de la distancia de la primera al hueco, \(\rho\).
		\begin{enumerate}
			\item Asumiendo que \(\theta\) no es necesariamente constante obtenga las ecuaciones de Lagrange para \(\rho\) e  \(y\). Resultado:\\[5 pt]
			\(M \left(- g + \ddot{y}\right) = \lambda_{1} \qquad m \left(- \rho \dot{\theta}^{2} + \ddot{\rho}\right) = \lambda_{1}\)
			\item Resuelva el sistema para \(\rho, y\) y el multiplicador de Lagrange \(\lambda_1\) encontrando las fuerzas de tensión sobre ambas masas.\\[5 pt]
			Resultado: \(Q_{\rho} = - \dfrac{M m \left(g + \rho \dot{\theta}^{2}\right)}{M + m}\)
		\end{enumerate}
	\end{minipage}
	\begin{minipage}[c][-1cm][t]{0.3\textwidth}
		\input{figs/linkedWeights.tikz}
		% \includegraphics[width=\textwidth]{cmchap6_fig6_5}
	\end{minipage}


	\item
	\begin{minipage}[t][4.5cm]{0.62\textwidth}
		\textbf{Partícula deslizando sobre una semiesfera} [Marion (e) ex. 7.10]\\
		La partícula de masa \(m\), considerada puntual, desliza sobre una semiesfera de radio \(R\) sin fricción.
		\begin{enumerate}
			\item Encuentre la fuerza de la ligadura.\\
			Resultado: \(Q_{\rho} = m \left(- R \dot{\theta}^{2} + g \cos{\left(\theta \right)}\right)\)
			\item Calcule el ángulo en que la partícula se despega de la semiesfera.\\
			Resultado: \(\theta^\mathrm{despegue} \approx 48.19^\circ\) 
		\end{enumerate}
	\end{minipage}
	\begin{minipage}[c][0cm][t]{0.3\textwidth}
		\input{figs/semiSphere.tikz}
		% \includegraphics[width=\textwidth]{marion7_10}
	\end{minipage}
	
	Para llegar al ángulo de despegue debe resolver la ecuación diferencial a la que arribará tras resolver la problemática de las fuerzas de ligadura, que será $\ddot{\theta} = \dfrac{g \sin(\theta)}{R}$.
	Esta expresión es integrable para el recorrido que hace la partícula.
	Para facilitar esto se intercala por regla de la cadena derivaciones en función de \(\theta\) en la definición de la aceleración.
	$$
		\ddot{\theta} 
		= \frac{d \dot{\theta} }{d t} 
		= \frac{d \theta}{d t} \frac{d \dot{\theta}}{d \theta} 
		= \dot{\theta} \frac{d \dot{\theta}}{d \theta}
	$$

	Como la partícula parte de $\theta(t=0) = 0$ con $\dot{\theta}(t=0) = 0$.
	$$
	\begin{aligned}
		\ddot{\theta} = \dot{\theta} \frac{d \dot{\theta}}{d \theta}
		&= \frac{g}{R} \sin(\theta)\\
		\dot{\theta} d \dot{\theta}
		&= \frac{g}{R} \sin(\theta) d \theta \\
		\int_0^{\dot{\theta}_\mathrm{despegue}} \dot{\theta} d \dot{\theta}
		&= \frac{g}{R} \int_0^{\theta_\mathrm{despegue}} \sin{\theta} d \theta\\
		\frac{\dot{\theta}^2}{2} \bigg|_0^{\dot{\theta}_\mathrm{despegue}}
		&= \frac{g}{R} (-\cos{\theta}) \bigg|_0^{\theta_\mathrm{despegue}}\\
		\frac{\dot{\theta}_\mathrm{despegue}^2}{2}
		&= \frac{g}{R} (-\cos(\theta_\mathrm{despegue}) + 1)\\
	\end{aligned}
	$$
	Con esto hay que substituir $\dot{\theta}^2$ en una expresión de $F^\mathrm{ligadura}_{\rho}$, que debe ser nula en el momento de despegue.


\end{enumerate}

\end{document}
