\documentclass[11pt, spanish, a4paper, twoside]{article}
% LTeX: language=es-AR

\input{../figurasLaTeX/introUNLaM}

\begin{document}
\begin{center}
  % \textsc{\large Mecánica general}\\
  \textsc{\large Fuerzas externas en el enfoque Lagrangiano}
\end{center}

\begin{enumerate}

\item 
%\textbf{MIT Pset ? ex ?} 
\begin{minipage}[t][8cm]{0.55\textwidth} 
\textbf{Barra que pende de un carro}\\
Obtenga las ecuaciones que describen la dinámica del sistema.
El momento de inercia para una barra de masa \(m\) y longitud \(l\) para una rotación desde uno de sus extremos es \(\frac{m}{12} l^2\). 
\begin{enumerate}
	\item Las fuerzas no conservativas que actúan sobre el sistema son el forzado externo \(\vec{F}(t)\), y la que hace ejerce amortiguador de constante \(b\) en función de la velocidad del carro,  \(- b \dot{x} \hat{x}\).
	Estas deben descomponerse en fuerzas generalizadas, pero primero obtenga las ecuaciones de Euler-Lagrange que correspondenderían a la dinámica si estas fuerzas no existieran.
	\item Ahora sí, descomponga en fuerzas generalizadas las no conservativas que actúan sobre el sistema.
	\item Calcule las ecuaciones de Euler-Lagrange con las fuerzas generalizadas.
\end{enumerate}
\end{minipage}
\begin{minipage}[c][-1cm][t]{0.4\textwidth}
	\includegraphics[width=\textwidth]{zweite}
\end{minipage}



\item 
%\textbf{MIT Pset 7 ex 2} 
\begin{minipage}[t][5.3cm]{0.6\textwidth}
\textbf{Péndulo de torsión desbalanceado}\\
Dos pesos de masa idéntica $m$ están unidos al extremo de brazos de masa despreciable.
Uno de los brazos describe una inclinación fija con la horizontal de $\phi$.
Descartamos la fricción con los rodamientos que mantiene vertical el eje de donde parten los brazos.
Este podría rotar libremente a cualquier ángulo $\theta$ si no fuera por un resorte de torsión de constante elástica $K_t$ que opone un torque buscando alinear la pieza con el plano x-z cada vez que $\theta \neq 0$.
Por tanto, se lo considera en la ecuación de Euler-Lagrange como una fuente de energía potencial elástica.
Pero, adicionalmente, se ejerce sobre el sistema otro torque, que es externo y es además variable en el tiempo: $\vec{\tau}= \tau (t) \hat{z}$.
\end{minipage}
\begin{minipage}[c][0cm][t]{0.35\textwidth}
	\includegraphics[width=\textwidth]{pset7ex2}
\end{minipage}
Pregunta conceptual:
¿Cuáles es la unidad de la fuerza generalizada?

\begin{enumerate*}[label=\alph*), itemjoin={\hspace{1.5cm}}]
  \item \si{\newton} 
	\item \si{\newton \over \metre}
	\item \si{\newton \metre}
	\item Otra
\end{enumerate*}

Despeje la aceleración angular de la ecuación de Euler-Lagrange. Resultado:
\(
	\ddot{\theta} = \dfrac{K_{T} \theta + \tau}{L^{2} m \left(\sin^{2}{\left(\phi \right)} - 2\right)}
\) 


%\newpage

\item
%\textbf{MIT Pset 7 ex 4} 
\begin{minipage}[t][7.5cm]{0.6\textwidth}
\textbf{Barriles soldados}\\
Dos barriles cilíndricos homogéneos de respectivas masas y radios $m_1$,$m_2$, $R_1$ y $R_2$ están soldados.
Este armado puede rotar en torno a su eje común que no le presenta fricción.
Una cuerda de masa despreciable envuelve al cilindro externo y sus extremos conectan un resorte de constante elástica $k$ y un amortiguador.
Tal amortiguador ejerce una fuerza de resistencia al movimiento lineal con la velocidad,
$$
\vec{F}_\mathrm{amortiguador} = - b \dot{\vec{r}}.
$$

Una correa de masa despreciable envuelve al cilindro de menor radio y de ella pende vertical un bloque de masa $m_o$.\\
Despeje la aceleración angular de la ecuación de la dinámica de Euler-Lagrange. 
Resultado:
\(
	\ddot{\theta} = \dfrac{2 \left(R_{1} g m_{0} - R_{2}^{2} b \dot{\theta} - R_{2}^{2} k \theta\right)}{2 R_{1}^{2} m_{0} + R_{1}^{2} m_{1} + R_{2}^{2} m_{2}}
\)
\end{minipage}
\begin{minipage}[c][-1cm][t]{0.35\textwidth}
	\includegraphics[width=\textwidth]{pset7ex4}
\end{minipage}



\item
%\textbf{MIT Pset 7 ex 6} 
\begin{minipage}[t][6cm]{0.5\textwidth}
\textbf{Plano inclinado oscilante}\\
Sobre la superficie inclinada en $\theta_0$ del carro de masa $m_0$ rueda sin deslizar un disco de radio $R$ y masa $m$.
Este no se sale de la superficie a pesar de que al centro del mismo se aplica una fuerza $\vec{F}= F(t) \hat{x}$ gracias a un resorte de constante elástica $K_1$ que une este centro con el carro.
Limita el alcance de este un resorte de constante elástica $K_2$ fijado a la pared y un amortiguador proporcional a la velocidad de constante proporcional $b$.
Ambos resortes tienen originalmente su longitud de equilibrio $l_{10}$ y $l_{20}$.
Se descarta la fricción del carro con el suelo.
Todo el sistema está sometido a la aceleración gravitatoria $\vec{g}= - g \hat{y}$.\\
\end{minipage}
\begin{minipage}[c][0cm][t]{0.45\textwidth}
	\includegraphics[width=\textwidth]{pset7ex6}
\end{minipage}
Pregunta conceptual: ¿Qué es la fuerza generalizada asociada al desplazamiento virtual $\delta x$ debida a $\vec{F}$?

%\begin{tasks}(4)
\begin{enumerate*}[label=\alph*), itemjoin={\hspace{1.5cm}}]
	\item $F(t) \cos(\theta)$
	% \task $F(t) \cos(\theta)$
	\item $F(t)$
	% \task $F(t)$
	\item $F(t) \delta x$
	% \task $F(t) \delta x$
	\item $0$
	%\task $0$
%\end{tasks}
\end{enumerate*}

Obtenga la dinámica a partir de las ecuaciones de Euler-Lagrange, esto es, haga explícita las aceleraciones de ambos cuerpos. Resultado:\\
\[
	\ddot{x} = \frac{2 K_{1} X_{1} \cos{\left(\theta_{0} \right)} - 3 K_{2} x - 3 b \dot{x} - g m \sin{\left(2 \theta_{0} \right)} - F \cos{\left(2 \theta_{0} \right)} + 2 F}{2 m \sin^{2}{\left(\theta_{0} \right)} + m + 3 m_{0}}
\]
\[
	\ddot{X}_{1} = \frac{2 \left(- K_{1} m X_{1} - K_{1} m_{0} X_{1} + K_{2} m x \cos{\left(\theta_{0} \right)} + b m \cos{\left(\theta_{0} \right)} \dot{x} + g m^{2} \sin{\left(\theta_{0} \right)} + g m m_{0} \sin{\left(\theta_{0} \right)} + m_{0} F \cos{\left(\theta_{0} \right)}\right)}{m \left(2 m \sin^{2}{\left(\theta_{0} \right)} + m + 3 m_{0}\right)}
\]



\end{enumerate}
\end{document}
