\documentclass[11pt, spanish, a4paper, twoside]{article}
% LTeX: language=es-AR

%% Asignación ejercicios
% 1, 2, 4 presente clase 
% 5, 6, 7, 8 en la siguiente clase


\input{../referencia/figurasLaTeX/introUNLaM}
% \usepackage{svg}

\begin{document}
\begin{center}
  % \textsc{\large Mecánica general}\\
  \textsc{\large Cuerpo rígido | Ecuaciones de Euler}
\end{center}

% De poder resolver estos problemas en forma autónoma puede asumir que adquirió los conocimientos mínimos sobre los temas abordados en la semana. No dude en consultar a docentes y compañeros si no puede terminarlos. Los problemas marcados con * son opcionales.

\begin{enumerate}

	\item 
	\begin{minipage}[t][5cm]{0.55\textwidth}
		\textbf{Engranaje inclinado}
		Un engranaje de masa de \SI{10}{\kilo\gram} está montado con una inclinación de \ang{10;;} en un eje de masa despreciable.
		Los cojinetes \(A\) y \(B\) sostienen el eje que gira con velocidad angular constante.
		El \(A\) es de empuje, por lo que provee reacción también en la dirección longitudinal al eje en tanto que el \(B\) solo lo hace en las direcciones transversales.
		Los momentos de inercia del engranaje son \(I_z = \SI{.1}{\kilo\gram\metre\squared}\) y el \(I_y = \SI{.05}{\kilo\gram\metre\squared}\).
		\begin{tasks} 
			\task Determine las reacciones que deben proveer los cojinetes para el instante en que el sistema en rotación presenta la disposición que se ilustra.
		\end{tasks}
	\end{minipage}
	\begin{minipage}[c][0.5cm][t]{0.4\textwidth}
		\includegraphics[width=\textwidth]{hibb_21-4}
		% \includegraphics[width=\textwidth]{hibb21_4}
	\end{minipage}


	\item 
	\begin{minipage}[t][4.6cm]{0.55\textwidth}
		\textbf{Volante de inercia}
		El volante de inercia centrado en \(G\) tiene una masa de \SI{10}{\kilo\gram} es solidario al eje de masa despreciable que gira con velocidad angular constante \(\omega_s= \SI{6}{\per\second} \) (radianes por segundo) sostenido por los cojinetes \(A\) y \(B\).
		El primero es de empuje, por lo que provee reacción también en la dirección longitudinal al eje en tanto que el segundo solo lo hace en las direcciones transversales.
		Un eje transversal al del volante sostiene la montura del cojinete \(A\) y también gira con velocidad angular constante \(\omega_p\).
		\begin{tasks}
			\task Determine las reacciones que proveen los cojinetes.
		\end{tasks}
	\end{minipage}
	\begin{minipage}[c][0.5cm][t]{0.4\textwidth}
		\includegraphics[width=\textwidth]{hibb_21-6}
		% \includesvg[width=\textwidth]{hibb_21-6.svg}
		% \includegraphics[width=\textwidth]{hibb21_6}
	\end{minipage}


	\item
	\begin{minipage}[t][3cm]{0.75\textwidth}
		\textbf{Rotación fuera de eje}
		Un cilindro homogéneo de masa \(m\), radio \(R\) y altura \(H\) gira con velocidad angular constante \(\vec{\omega}\) en torno a un eje que forma un ángulo de \ang{30;;} con el \(\hat{z}\) y que pasa por su centro de masa.
		\begin{tasks}
			\task Calcular el torque que debe aplicarse al cilindro para mantener tal movimiento.\\
			Resultado: \(
			\vec{\tau} = \left[\begin{matrix}\frac{\sqrt{3} m \omega^{2} \left(- H^{2} + 3 R^{2}\right)}{48}\\0\\0\end{matrix}\right]
			\) 
		\end{tasks}
	\end{minipage}
	\begin{minipage}[c][2cm][t]{0.2\textwidth}
		\includegraphics[width=\textwidth]{Ex3_17}
	\end{minipage}



	\item 
	\begin{minipage}[t][4.5cm]{0.75\textwidth}
		\textbf{Barra en rotación}
		La barra delgada que se extiende desde A hasta B tiene una masa \(m\) y está conectada al soporte por medio de un pasador en A.
		El soporte está rígidamente montado en la flecha.
		Determine la velocidad angular constante requerida \(\omega\) de la flecha, para que la barra forme un ángulo \(\theta\) con la vertical.
	\end{minipage}
	\begin{minipage}[c][3cm][t]{0.2\textwidth}
		\includegraphics[width=\textwidth]{hibb_21-45}
	\end{minipage}


	\item 
	\begin{minipage}[t][3.5cm]{0.6\textwidth}
	\textbf{Cilindro desbalanceado}
		Un cilindro de altura \(D\) y masa \(M\) gira apoyado en dos cojinetes \(P\) y \(Q\) con velocidad angular \(\omega\).
		En un eje imaginario en un ángulo \(\varphi\) del eje de rotación, y a una distancia \(a\) de su centro, tiene colocadas dos pesas de igual masa, \(m\). 
		\begin{tasks} 
			\task Calcular la fuerza que aplican los cojinetes.\\
			Resultado: \(
				F = \frac{2 m a^2 \omega^2}{D} \sin(\varphi) \cos(\varphi)
			\).
		\end{tasks}
	\end{minipage}
	\begin{minipage}[c][0.5cm][t]{0.35\textwidth}
		\includegraphics[width=\textwidth]{cilindroDesbalanceado}
	\end{minipage}
		

	\item 
	\begin{minipage}[t][4.5cm]{0.55\textwidth}
		\textbf{``Flecha'' sobre cojinetes}
		La flecha se construyó con una barra cuya masa por unidad de longitud es de \SI{2}{\kilo\gram\per\metre}.
		Determine las componentes \(x, y, z\) de la reacción en los cojinetes A y B si en el instante que se muestra la flecha gira libremente a una velocidad angular de \(\omega = \SI{30}{\per\second}\) (radianes por segundo).
		¿Cuál es la aceleración angular de la flecha en este instante?
		El cojinete A es capaz de soportar una componente de fuerza en la dirección y mientras que el cojinete B no.
		% Ignore la masa del eje.
	\end{minipage}
	\begin{minipage}[c][0cm][t]{0.4\textwidth}
		\includegraphics[width=\textwidth]{hibb_21-48}
	\end{minipage}


	\item 
	\begin{minipage}[t][6cm]{0.65\textwidth}
		\textbf{Aceleración angular constante}
		Un cilindro de 15 lb gira alrededor de \(\overline{AB}\) con una velocidad angular \(\vec{\omega} = -\SI{4}{\per\second} \hat{x}\).
		El cojinete \(A\) no soporta fuerza en la dirección de \(x\); dicha función la cumple exclusivamente el cojinete \(B\).
		Además, el soporte apoyado en \(C\) rota con una aceleración angular \(\vec{\alpha}_C = \dot{\vec{\omega}} = \SI{12}{\per\second\squared} \hat{Z}\), donde \(\hat{Z}\) es paralelo a \(\hat{z}\) pero alineado a \(\overline{CA}\).

		\textbf{Ayuda:}
		Para que el centro de masa \(G\) del cilindro describa un movimiento circular a una distancia \(d_{AG}\) del cojinete \(A\), es común introducir una \textbf{aceleración centrípeta}.
		En este caso dicha aceleración se expresa como \(\vec{a}_{G} = \dfrac{|\alpha_c|}{d_{AG}} \hat{x}\).
		Este término surge porque, al girar junto con el cilindro, el sistema de referencia se convierte en un \textbf{sistema no inercial}, en el cual se percibe una \textbf{fuerza ficticia centrífuga} dada por \(- m \dfrac{|\alpha_c|}{d_{AG}} \hat{x}\).
	\end{minipage}
	\begin{minipage}[c][2cm][t]{0.3\textwidth}
		\includegraphics[width=\textwidth]{hibbEng_21-58}
		% \includegraphics[width=\textwidth]{hibb_14_21_58}
	\end{minipage}
	Tanto al incluir explícitamente una aceleración centrípeta en las ecuaciones de la segunda ley de Newton como al considerar la fuerza ficticia centrífuga, se obtiene el mismo resultado físico.
	No obstante, es fundamental recordar que estos artificios matemáticos son consecuencia de trabajar en un sistema de referencia no inercial.
	\begin{tasks}
		\task Convierta los datos en unidades imperiales (pies, libras) en unidades del Sistema Internacional.
		\task Determine las reacciones que deben proveer los cojinetes.\\
		Resultado: \(
			\left[\begin{matrix}A_{y}\\A_{z}\\B_{x}\\B_{y}\\B_{z}\end{matrix}\right] = \left[\begin{matrix}-7.52\\33.4\\268.0\\-17.4\\33.4\end{matrix}\right]
		\)
		en unidades del SI.
	\end{tasks}


	\item 
	\begin{minipage}[t][3.5cm]{0.65\textwidth}
		\textbf{Trituradora de roca}
		Una trituradora de roca se compone de un disco delgado grande el cual está conectado por medio de un pasador a un eje horizontal.
		Si este gira a una velocidad constante de \SI{8}{\per\second} (radianes por segundo), determine la fuerza normal que el disco ejerce en las piedras.
		Suponga que el disco rueda sin deslizarse y que su masa es de \SI{25}{\kilo\gram}.
		Ignore la masa del eje.
	\end{minipage}
	\begin{minipage}[c][2cm][t]{0.3\textwidth}
		\includegraphics[width=\textwidth]{hibb_21-56}
	\end{minipage}


\end{enumerate}

\end{document}
